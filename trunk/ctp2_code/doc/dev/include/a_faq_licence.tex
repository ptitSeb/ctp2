% Last update (Last post): 06.04.2004 14:45:47
% SRC: http://apolyton.net/forums/showthread.php?postid=2873946#post2873946
\begin{section}{Licence FAQ}
\addquestion{Is the CtP2 source code Open Source?}{
No, it is not. The End User License Agreement (which can be found \href{http://apolyton.net/csd.php}{here}) has a number of restrictions on it that prevent it from being called 'open source' software by the common and most widely accepted definitions of that term. Most importantly, for an open source product, the distribution of the code should be entirely free, while the the EULA of the CtP2 source code doesn't allow distribution without written permission from Activision, or any kind of commericial distribution. You can find the industry standard for an Open Source Definition on the Open Source Initiative (OSI) \href{http://www.opensource.org/docs/definition_plain.php}{website}.\\
\\
Note that the text of the EULA and Activision's practical interpretation of it seem to be rather different, the interpretation Activision chooses (as became clear from contact with them) seems to be far less restrictive than the official EULA. Of course, in case of doubt it is advisable to refer to the EULA for reference, as "Activision's interpretation", though useful for daily affairs, offers little to no legal basis. The rest of this FAQ will where possible explain both the official EULA and Activision's interpretation of it (or as what I perceive to be their interpretation -- to make things even more ambiguous).
}

\addquestion{Am I allowed to distribute the source?}{
The EULA forbids any kind of distribution of the original source code without prior written consent by Activision. Also, under no conditions whatsoever are you allowed to in any way distribute the source commercially.\\
\\
After communicating about this with them, Activision said their interpretation is that it's okay to distribute the source, as long as you make sure that anyone who has access to it must agree to the EULA first. For the standard exe installer this applies, as the installer itself has an EULA screen that you must agree to. If (parts of) the source is (are) distributed in any other format than with the original exe installer (e.g. in zip format), some kind of other mechanism must be in place to ensure the user agrees to the EULA before he can gain access to the code. For commercial distribution of code, one should contact Activision to obtain prior written permisison (contact info is in the EULA). All of this applies to both the original source code and any modified source code.
}

\addquestion{Am I allowed to distribute modified versions of the game?}{
Note: this question is about playable, compiled versions of the game. For uncompiled code, see the previous issue.\\
\\
According to the EULA, you are indeed allowed to distribute new versions, as long as you do not distribute them commercially, label them clearly as a non-Activision product, don't include illegal/obscene/privacy-sensitive infrormation and as long as they require the retail product to function.\\
\\
Activision says that distribution of new updates are okay, as long as they require the original game to function and all other EULA conditions apply (e.g. regarding labeling the opening screen, illegal/private content, being free of charge, etc).
}

\addquestion{Am I allowed to use a CVS server to develop the game?}{
The EULA forbids keeping copies of the source code for reasons other than backup. Using a CVS server would require you to keep a copy of the code for another reason (i.e. to coordinate the work of several people), so is therefore strictly speaking not allowed.\\
\\
Activision's interpretation is that the use of CVS servers and similar tools is allowed, as long as anyone who has access to the code has agreed to the EULA.
}

\addquestion{Is it allowed to port the game to other platforms (Linux, Mac, Amiga)?}{
The EULA isn't very specific in this regard, but seems to allow it as long as you make sure the retail product is still required to play the ported version.\\
\\
Activision's interpretation is that it is certainly allowed, as long as you make sure the retail product is still required to play the ported version.
}

\addquestion{How can I legally make files available for download in this forum?}{
For your own creations you can just post them as you would with any files. For the CtP2 source code and materials originating from it, there are basically two situations:\\
\\
1. Files that contains (portions of) the source code. Only people who've agreed to the EULA are allowed to have access to these. To make sure of that, we created a little script: if you want to make any source code files available for download, make sure the URL has the format http://apolyton.net/csd.php?\{url\}, where \{url\} is the actual URL address of the file (Example:
 \href{http://apolyton.net/csd.php?http://apolyton.net/ctp2/files/CTP2_Source.exe}{http://apolyton.net/csd.php?http://apolyton.net/ctp2/files/CTP2\_Source.exe}
  -- right-click and select properties to see the URL; left-click the link to see the effect). This will force anyone who wants to download your source code files to agree to the EULA first. (Note: disable 'automatically parse URLs' in the reply screen if you're having difficulties getting the link right.)\\
\\
2. Actual executables to run the game. You can just upload these like other files, but these do require the warning at start-up and in the documentation that they're not Activision material and author \& email info, as the EULA specifies. At least for the Apolyton project you can simly use 'Apolyton' or 'Apolyton CtP2 Source Code Project' or similar as author, and for email address you can use \href{mailto:ctp2source@apolyton.net}{ctp2source@apolyton.net} (we created that address for this purpose). You can either use an message box to display this error, or replace upsg001.tga with a modified version (or use yet another solution). See \href{http://apolyton.net/forums/showthread.php?postid=2463088\#post2463088}{here} for two practical implementatons.
}
\end{section}% Licence FAQ
